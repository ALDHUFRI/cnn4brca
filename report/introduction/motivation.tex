Breast cancer is the most commonly diagnosed cancer in woman and its death rates are among the highest of any cancer. It is estimated that about 1 in 8 U.S. women will be diagnosed with breast cancer at some point in their lifetime. Early detection is key in reducing the number of deaths; detection in its earlier stage (\textit{in situ}) increases the survival rate to virtually 100\%~\cite{Howlader2014}.

With current technology, a high quality mammogram is ``the most efective way to detect breast cancer early''~\cite{Mammograms2014}. Mammograms are used by radiologists to search for early signs of cancer such as tumors or microcalcifications. About 85\% of breast cancers can be detected with a screening mammogram~\cite{PerformanceMammography2013}; this high sensitivity is the product of the careful examination of experienced radiologists. A computer-aided diagnosis tool (CAD) could automatically detect these abnormalities saving the time and training needed by radiologists and avoiding any human error. Computer based approaches could also be used by radiologists as a help during the screening proccess or as a second informed opinion on a diagnosis.

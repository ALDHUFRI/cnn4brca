%[Accurately] [discriminating|distinguishing|separating] between cancerous and normal [breast] tissue in [radiographic images of the breast|mammographic images] is a complex problem in [Medical] Image Analysis[;in|.In] this thesis, we use [deep learning techniques, particularly] convolutional networks to tackle this problem. [We [show|prove] that [convolutional networks|this model] can be succesfully applied to these kind of problems. Furthermore, we present a first result in the segmentation of breast masses in digital mammograms.]

Discriminating between cancerous and normal tissue in radiographic images of the breast is a complex problem in medical image analysis; in this thesis, we use convolutional networks to tackle it.

%[Breast] Cancer is caused by abnormal cells [that [grow|divide]|growing|dividing] [out of control|uncontrollably|unconstrainedly|unrestrictedly][, form tumors and invade surrounding tissue| [eventually] forming tumors and [eventually] invading surrounding tissue].
Cancer is caused by abnormal cells dividing uncontrollably, forming tumors and eventually invading surrounding tissue. It receives different names based on the part of the body where it originates. 
Breast cancer, among all cancers, has the highest incidence rate in the United States, an estimated 14.1\% of cancer diagnoses in 2015, and the third highest mortality accounting for 6.9\% of all cancer-related deaths. 
Among women, it is the most commonly diagnosed---29\% of all cancer cases---and, besides lung cancer, the deadliest---killing 15\% of all diagnosed cases~\cite{ACS2015}. The American Cancer Society recommends women aged 45 or older to get mammograms, images of the breast that show signs of tumor formation, annually or biennially~\cite{Oeffinger2015}. We consider two types of diagnostic lesions detected on mammograms: clustered microcalcifications, tiny deposits of calcium that could appear around cancerous tissue; and breast masses, more direct signs of the existence of a tumor, although often benign.
%The quality of a mamogram and the diligence and experience of the radiologist is an important factor to succesfully detect breast cancer.

%Although radiologists are able to [accurately] identify these lesions with high [accuracy/sensitivity], [computerized|automatic] examination may be used: to direct their attention to [relevant|dangerous|critical|crucial] regions, as a second informed opinion or when doctors are unavailable.
Although radiologists are able to identify these lesions with high accuracy, computerized examination may be used to direct their attention to relevant regions, as a second informed opinion or when doctors are unavailable. 
%This motivated the [research group|department] to [create a project to] design a computer-aided diagnosis system (CAD) for breast cancer [detection|diagnosis][;|.] [the present thesis| this thesis] falls under the scope of this project as its first attempt to use deep learning for [breast cancer diagnosis|lesion segmentation].
This motivated the research group to design a computer-aided diagnosis system (CAD) for breast cancer; the present thesis falls under the scope of this project as its first attempt to use deep learning for lesion segmentation.

%Traditional CAD systems [work as|are] pipelines that process the image sequentially using different computer vision [and machine learning] techniques [at each stage].
Traditional CAD systems are pipelines that process the image sequentially using different computer vision techniques; for instance, an standard layout will preprocess the image, identify regions of interest, extract features from the relevant parts and train a classifier on the extracted features.
%Although many successful systems are built [following this pattern|in this fashion], [it|they] presents [a couple of|a few|two] disadvantages: each stage uses intricate algorithms and handcrafted features [making it brittle|creating [a complex and brittle|an overly complex] system] and who need to be properly tuned by expertsstages are dependent and the relations between them are often obscure: changes in one component [affect|influence] [the performance of] other parts of the system, [all components| every component] [(and its algorithm)] needs to [have a high performance|perform [properly|well|superbly|highly]] for the [overall] system to [perform well | have a [high|good] performance], each component needs to be improved to improve overall results;; among others.
Although many successful systems are built following this pattern, it presents two disadvantages: (1)~stages use intricate algorithms and handcrafted features creating an overly complex system that requires many experts to be modified or properly tuned and (2)~stages are dependent and relations between them are often obscure: changes in one component affect the performance of others, every component needs to perform well for the system to perform well and every component needs to be improved to improve overall performance.

We use convolutional networks to replace most, if not all, of the stages of traditional image processing. 
%Convolutional networks~\cite{Fukushima1980,LeCun1998}, a natural extension to feedforward neural networks, are [a] statistical learning [classifier|models] that use raw images as input and learn the important [image] features for the classification [task] during training. 
Convolutional networks~\cite{Fukushima1980,LeCun1998}, a natural extension to feedforward neural networks, are statistical learning models that use raw images as input and learn the relevant image features during training. They work well with minimally preprocessed images and encapsulate segmentation, feature extraction and classification in a single trainable model. Despite some drawbacks, convolutional networks are the state-of-the-art technology for object recognition~\cite{Russakovsky2015}.

Researchers have used small convolutional networks to separate breast masses from normal tissue~\cite{Sahiner1996} and individual microcalcifications from noise in the image~\cite{Lo1995, Ge2007}. A bigger network incorporating newer features such as rectified linear unit activations, max-pooling, momentum, data augmentation and dropout was trained to identify malignant masses~\cite{Arevalo2015}. In these experiments mammograms were enhanced, potential lesions were located and relevant regions were presented to the network for classification.
Recently, Dubovrina et al.~\cite{Dubrovina2015}, segmented different breast tissue using a modern convolutional network. This work relates closely to our own. We train deep convolutional networks end-to-end to identify lesions in digital mammograms.%We train deep convolutional networks end-to-end to identify lesions in digital mammograms.

%We start our experiments by training a simple convolutional network to detect masses and clustered microcalcifications, later we train a more complex network architecture including some of the most recent advances and finally use the gathered knowledge to build an optimal convolutional network with tuned hyperparameter. As further experiments and depending on the available time we plan to pretrain a convolutional network with a different image database and fine-tune it using our database, use an all convolutional architecture, account for the problem of using an unbalanced data set and use an ensemble of networks. 
We aim to learn whether convolutional networks could automatically segment mammographic images, how advantageous it is to use a bigger architecture, more data and tuned hyperparameters and whether we can achieve results similar to those of traditional systems.

In this introductory chapter, we emphasize the importance of the problem in Section~\ref{sec:Motivation}, expand into the problem with traditional image analysis methods in Section~\ref{sec:Problem}, expose the particular objectives and hypotheses of the thesis in Sections~\ref{sec:Objectives} and~\ref{sec:Hypothesis}, offer a brief summary of our methodology in Section~\ref{sec:Methodology}, highlight the particular contributions of this work in Section~\ref{sec:Contributions} and, lastly, offer an outline of the thesis in Section~\ref{sec:Outline}.

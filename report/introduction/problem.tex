Image segmentation partitions an image into multiple regions, essentially assigning a label to each pixel. For instance, classifying each pixel in a street image as road, building, tree, pedestrian or bicycle. If we are only interested in discriminating between a single class and the background, we could also present
we could also present a heatmap with the different probabilities in each pixel of the image, for instance on mammograms we could present a grayscale image of whether a lesion is present.
In medical imaging, automatic lesion segmentation is tasked with distinguishing lesions from normal tissue.
% doctors need to identify in each image if there are any lesiona and wheter they are malignant, this is done by scaning the mammogram looking for abnormal findings. this is error prone and its efficacy depends greatly on the preparation and dilligence of the doctor. 

CAD systems are based on image and classification techniques coming from Artificial Intelligence and Machine Learning. Traditional CAD tools for breast cancer detection are composed of three steps: feature extraction, feature selection and classification. In the feature extraction phase, the system uses filters and image transformations to preprocess the mammogram and find geometric patterns, producing a set of features for the image; expert knowledge is often needed in this phase. Feature selection or regularization is used to focus only on the important features for the classification task. Once a vector of features is obtained for each image, an standard binary classifier performs the final detection or diagnosis. These techniques have been used for many years and are standard in the industry~\footnote{See~\cite{Hernandez2014} for an example of a CAD system developed in this institution.}. 
%traditional cad systems are ill-defined to deal with this difficult problem subpar

Despite its widespread use and efficiency, systems based on traditional computer vision techniques have various limitations that should be addressed to further improve its performance:
\begin{itemize}
	\item There is no standard way of preprocessing mammograms. Some techniques are commonly used but their performances can vary.
	\item It uses handcrafted features. The features extracted from the image are chosen beforehand (maybe designed with the help of experts) and special filters and image techniques are used to extract them.
	\item Segmentation and image feature extraction are error-prone and could greatly affect the classification results.
	\item It normally uses a small patch of the mammogram and makes a prediction on that patch but it does not consider the entire mammogram neither to make a prediction on the patient or to account for correlation between patches.
	\item To produce good results it requires knowledge in various fields such as radiology, oncology, image processing, computer vision, machine learning, etc.
	\item It is composed of many sequential steps. At each stage, there are many techniques from which the researcher can choose and many parameters which have to be estimated. This represents a cost in time and results as it is improbable that the optimal selection of techniques and parameters is achieved.
	\item As it is a complex system with different subsystems involved many other issues can arise such as non desired or unknown dependencies between subsystems, difficulty to localize errors, maintainability, etc.  
	\item The techniques currently used are complex but the improvements achieved are not substantial. Much work is needed to make only incremental improvements and it is hard to know to which part of the system dedicate more resources.
\end{itemize}

% To remedy this I plan to do this, i do image segmentation with convnets. end-to-end,
This project will center around using Convolutional Networks, a recent development in computer vision, (see Section~\ref{subsec:ConvNets}) to tackle some of these limitations, especifically automate preprocessing, feature extraction and segementation, use entire mammogram images and simplify the system pipeline by using a convolutional network as a replacement for many steps traditionally performed in succesion.

We document here the decisions taken to obtain the small image patches $x$ and its respective labels $y$ from the chosen databases.

To obtain these image patches from the entire mammogram we move a square window across the image similarly to the way convolutional filters move accross an image and store the image patch directly beneath it. This generates a big number of small patches from each mammogram and takes advantage of the translational invariance of our data, i.e., a breast mass will continue to be a breast mass no matter its position in the image. The size of these patches, its labels and the overlapping between them depend on some variables that we define next.

\paragraph{Image dimensions}
We use square images because they are common in practice and simplify data augmentation. To define the size we have to consider two aspects: keeping a manageable input size for the network (in pixels) and capturing the entire lesion in the image (in mm).

The smallest microcalcification worth considering could be as small as 0.16 mm~\cite{Lo1998}, thus the spatial resolution should be at most 0.16 mm. The standard definition of a cluster of microcalcifications is of 5 or more inside a 1 $cm^2$ area~\cite{Sickles2013}, thus the entire image patch should cover at least a 1 $cm^2$ area. Using a spatial resolution of 0.16 mm and an image size of $64 \times 64$ pixels we cover an area of 1.024 cm $\times$ 1.024 cm = $\sim$1.05 $cm^2$.
% Pixel size .15625 mm i get a 10 mm2 area

Mass sizes (length of the long axis) vary from 5 mm to 20 mm~\cite{Sahiner1996}~\footnote{Bigger masses are easily detectable by touch and thus less important for our purposes.} There is not really any restriction on spatial resolution other than it being good enough to capture texture information. Using a spatial resolution of 0.32 mm and an image size of $64 \times 64$ pixels we cover an area of 2.048 cm $\times$ 2.048 cm = $\sim$4.2 $cm^2$.
% pixel size .3125 equals area 20 mm
%pixel size 0.39 equals area 25 mm
% pixel size .4 mm equals ara 25.6 mm (exactly the size in Sahiner1996)

Although we use the same input size ($64 \times 64$) for microcalcifications and masses they do not cover the same area in the mammogram. We need to use two different sizes because if we preserve the spatial resolution of 0.16 mm, the 1 $cm^2$ area would not be able to contain the entire mass meanwhile if we use a spatial resolution of 0.32 mm, some microcalcifications will dissapear and the 4 $cm^2$ area would have way too much noise compared to the size of the cluster of microcalcifications.

An alternative is to use a $128 \times 128$ pixels image patch with 0.16 mm, this will result on the same 4 $cm^2$ area needed for masses with the spatial resolution needed for microcalcifications allowing us to train a single network for both kinds of lessions. Nonetheless, this has some critical flaws: the number of learnable parameters will almost double, the GPU may run into memory bottlenecks because of the increased number of parameters and unnnecesary details (noise) will be included in the image.

\paragraph{Stride}
% Size of the image to see how many strides can i Give.
Depends on the size of the lession, if it is way too big i may miss some lession. for instance a small microcalcification cluster
and how much overlap we want from image to image.
or how many images fdo we obtain from an entire mammogram.
More data augmentation if less stride.

\paragraph{Padding}
Should I use some padding so that I don't lose lesions which are in the very corner. For example, if I use a 2.5cm square and in there is a 1$cm$ mass close to the end of the image, then it may not detect it. Maybe not, I don't think ther eis going to be that many images on the side.

\paragraph{Labeling}
When the lesion hits the middle of the image, when the lession overlaps with the image or when the iomage is a given porcentage of the lession, or when a given porcentage of the lession apears in the image

\paragraph{Label info}
We will only use mass, MCC, normal, benign, malign and nothing.

\paragraph{Additional label information}
Is there any other info needed?. age breast density per case. per abnormality, assesment, subtlety BI-RADs words,  

\paragraph{Image enhancement}
I will cut the images first and store them as is. Later truy different enhancements per image
Contrast after cropping, or contrast before cropping.
\paragraph{Data cleaning}
Do I need to remove the marks and arrows and thingys. Could I let the network learn that those are not microcalcifications

\paragraph{data augmentation}
Should I do the image enhancement and data augmentation(rotations and flipping) during training or before hand. Or only the enhancement beforehand. Can I do them withouth changing the label? (depends on how the label is assigned, rotations and mirrors leve the same four squares in the middle of the image, thus, the label should not change, sclaing and others will.). Does it affect to present all different augmentations of the same image in one batch rather than in different batches

\paragraph{whjat about the blakc spaces}
Should I remove the black spaces or images that are more than 50\% black or something like that. Should I do it during this stage or after the cut. If i do what is going to be thenetwork performance when presneted with an entirely black input. what about lession which are pressed agains the breats skin, if i delete these images, they may dissapear.
40\% black out.

\paragraph{Resizing}
Does it affect the quality of the image what kind of resizing I do, should I use interpolation resizing or something. Hopefully I will always have to downsample so I may not lose much.

\paragraph{Total number of image patches.}

\paragraph{Resukting data base}
Generate a data set like the ImageNet challenge/


\begin{comment}
Elaborar el {\it Marco Teórico} consiste en sustentar teóricamente el estudio.
 Ello
implica una revisión bibliográfica del estado del arte, la construcción de una
perspectiva teórica, las técnicas, las investigaciones y antecedentes
relacionados con el problema y que establecen los fundamentos para llevar a
cabo la investigación. \\

La ecuación \ref{eqejem} muestra un ejemplo de {\bf expresión matemática:}


\begin{equation}
 \label{eqejem}
  \begin{array}{c}
    k_1f(x_n, y_n) + k_2\left[ f(x_n, y_n) + \alpha
      hf_x(x_n,y_n) + \beta hf(x_n,y_n)f_y(x_n,y_n) \right] \\
    = f(x_n,y_n) + \frac{h}{2}\left[
      f(x_n,y_n)f_y(x_n,y_n) \right]
  \end{array}
\end{equation}
\end{comment}



\begin{comment}

describe f(x)

aartificial neural networks
BISHOP ann 
non linea classifiers mad eof layers of neurons or processing untis which make a separte computation and have paramteres which need to be learned. 
When using sigmoids it models the probability of ocurrence of a given event. 
they come from 1960
A n artificial neruon has a set iof inputs representative of the dendrites on treal neurons, an output  that represents the axon and an activation funciton (traditionally a sigmoid function) that represents the computation or preference of a given neuron.
Layers

Trained via backpropagation



Convolutional networks 

Feedforward networks (left) initially proposed as two-layer perceptrons (Rosenblatt, 1958) and later developed into “deep” networks (e.g., Hinton et al., 2006 and LeCun et al., 1998). 
Convolutional networks, also known as ConvNets or CNNs, were first introduced by Fukushima et al. )=sskks and later refined and exapnded by LeCun)=. The first convolutional networks, called neocognitron (from neocortex and pereptron) are a natural expansion of simple Artificial Neural  Networks inspied on the way our neocortex process visual information.
Each neuron does that and that, actrivation function, figure. this neurons were used on  this and this.
Lecun workes on this and this,.. adding that and that and showing its applicability in that and that. 
More recently, various advances have been fostered by the interest on deep learning and more computational power as... dropout, ReLus, ...

GPU gains and recent advances have nmad econvnets one of the leading options for o ject recognition tasks in visual image recognition as imageNet, ...
They have replaced standard methods that require handcrafted feature selection on various subjects

although, convnets were inspired on neocortex they ghave bracehd out of this initially simple convolutional netowkrs and become veruy different in aras of finding improvemest over tye classification and without regard of the biology behiund oit.thus, they do not represent a valid model of the ewya the human braoin processes visual information. Reagardless of that, some davances in neuronscience still inspire new advancement s in convnets and some of the onvnets results ahave shown some results in neuroscience to be true (cite...)

Training on backpropagation, 

side note on language
I will sometimes refer to convolutional netowkrs with no pooling or maxout layers as simple convolutional networks and to those developed more recently simply as convolutional networks. i would also put the parameters of each onvnet to avoid any confusion
also, even thpough Although the convolutinoal nmetworks are an artificial neutral network AND THUS INSPIRED INTITIALYY ON THE BIOLOGY OF NEURAL NETWORKS, on this work I would USE THE NMORE STANDAR NAMES OVER THE BIOINSPIRED ONES, THUS prefer to use more standard names to defoine its parts over the more bioinspireds, thjus, units over neurons and convolutional networks over convolutiona networks. Other authors prefer the more bioinspired terms and both could be find interchangaebky in the literature



Mammogram databases
Description of what we have?

\end{comment}

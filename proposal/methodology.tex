In order to achieve the proposed objectives and test our hypotheses we will need to carry out various tasks. We list them here in the order in which we plan to execute them:

\begin{itemize}
	\item Literature review: A thorough review of the published work using the databases and resources available in the institution. By the end of this task, a complete theoretical background should be obtained and written. This will also help refine the scope of the project and the experiments to be conducted.
	\item Software review: Once a clear idea of what are the possible experiments to be executed, we will need to find appropiate software to perform them. Software for database managing, preprocessing and implementation of different neural networks should be either located or developed.
	\item Database preprocessing: We will ready the database images for the experiments; these implies joining different databases, obtaining the required features, preprocessing the images, assigning labels, etc.
	\item Assesing image preprocessing: We will train a standard convolutional network with fixed parameters on mammograms with three different preprocessings: no preprocessing, image enhancement using median or gaussian filters and wavelet filtered images. Furthermore, we will train a deeper convolutional network on nonpreprocessed images. We want to answer three research questions: which is the best preprocessing for convolutional networks, is using the best filter significantly better than using nonpreprocessed images and can a convolutional network automatically preprocess the images?
% Q: Is it better to make different preprocessings oin the same convolutional network or to fit each convolutional network for each preprocessing, thus, giving it the best chance to perform but taking more time.
	\item Exploratory experiments: We will train standard convolutional networks in two different inputs: small image patches obtained from mammograms and whole mammogram images. We will also train a linear classifier, probably rectified linear units, on the features obtained from a convolutional network trained on the ImageNet database, i.e., we will use an already trained convolutional network instead of one trained specifically in mammograms. Here we will use the image preprocessing technique that showed better results in the previous step. We want to answer two research questions: Can a convolutional network trained on whole mammograms perform as well as one trained on small patches and can we use an already trained convolutional network to classify mammograms?
	\item Model selection: Using the insights from previous sections and the current literature on convolutional networks, we will select a network architecture along with novel features, preprocessing, training and regularization procedures. We aspire to find the best convolutional network configuration for mammogram classification.
	\item Further experiments: We will train the chosen convolutional network on our mammographic database. We will perform crossvalidation to adjust the most important network parameters and use regularization to avoid possible overfitting. We want to answer two research questions: is the performance of the convolutional network considerably improved by parameter tuning and, more importantly, is this a good performance?.
%Maybe train one with no tune fitting.
	\item Gathering results: Produce results on the test set and elaborate figures and tables. This could be obtained directly from software output or from further program executions.
	\item Reporting results: Write the thesis and any article or technical guide which may result from this work. Both this and the previous step will be performed along the execution of the project, hopefully benefiting from the supervisors' feedback.
\end{itemize}
Finally, we want to note that this is an idealized workflow and some changes may occur due to time limitations or resources unavailability. In the unlikely case that the work is finished before the project deadline, we will either reiterate on model selection, experiments, result gathering and reporting or look into digital tomosynthesis, network ensembles or evolving convolutional networks.

\begin{comment}
La {\it Metodología} (o lo que algunos autores llaman el {\it Método})
 es el proceso o
conjunto de pasos que debe efectuarse para llegar a cumplir con los
objetivos. Esos pasos deben contener  los experimentos a realizar, la forma de
llevarlos a cabo, la evaluación de los resultados, la prueba de las hipótesis,
la respuesta a las preguntas de investigación y el último paso debe ser el
reporte escrito de los resultados.
\end{comment}

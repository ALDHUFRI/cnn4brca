%\emph{Machine learning} is the study of algorithms that create models (either of a population or of a function of interest) and estimate their parameters from data  in order to make good predictions or inferences.
%Machine learning is the study of mathematical models either of a population or of a function of interest and the algorithms used to estimate their parameters from data in order to make predictions or inferences.
\emph{Machine learning} is the study of algorithms which build models of a population or a function of interest and estimate their parameters from data in order to make predictions or inferences. A machine learning expert knows how to choose the right model for the problem in hand (\emph{model selection}), how to efficiently estimate its parameters from the available data (\emph{learning} or \emph{training phase}) and how to evaluate the trained model (\emph{testing phase}).

Machine learning problems can be divided into three categories depending on the data used to train the model: \emph{supervised learning}, where we learn a function $f(x)$ using a set of examples which are labelled with the correct output, for instance, learning a function that estimates the price of a house given its size and number of bedrooms from a dataset of houses labelled with their real value; \emph{unsupervised learning}, where we look for relationships and structure in unlabelled data, for instance, given a dataset of potential customers find those who are likely to buy a car and \emph{reinforcement learning}, where the only feedback received are rewards, for example, learning to play tetris from a dataset of world states and actions and where rewards are received sparsely every time points are earned (when lines dissapear). Supervised learning can be further divided in regression and classification. If the expected output is numerical, e.g., the price of a house, it is called \emph{regression}, if the expected ouput is categorical, e.g., spam or no spam, it is called \emph{classification}. We will focus on classification.

A \emph{classifier} takes as input a vector of \emph{features} $x \in \mathbb{R}^n$ from a problem instance and produces an \emph{output} $h(x)$ predicting the class $y$ that instance belongs to, i.e., it concretely models the underlying function $f(x)$ as $h(x)$ ($h$ stands for hypothesis). \emph{Binary classification}, when $y$ can only take two values e.g., cancer/no cancer, is the most common kind of classification and \emph{multiclass classification}, when $y$ can take $K > 2$ different values, can be performed by using $K$ binary classifiers. Some classifiers, such as convolutional networks (defined in Section~\ref{subsec:ConvNets}, output a \emph{score vector} $h(x) \in \mathbb{R}^K$ where $h(x)_k$ is a measure of the probability that $x$ belongs to class $k$. Every classifier partitions the \emph{feature space}, the $n$-dimensional space where features exist, into separate \emph{decision regions}, regions of the space which are assigned the same predicted outcome; a \emph{decision boundary} is the hypersurface that partitions the feature space. Classifiers are sometimes classified as \emph{linear} or \emph{nonlinear} according to the nature of the decision boundary they impose on the feature space. Logistic regression, for instance, is a linear classifier while an artificial neural network (with at least one hidden layer) is nonlinear.
% A linear classifier can perfectly separate linear data, while for nonlinear data a more powerful classifier is needed. Linearly separable data are those which can be classified by a linear classifier while nonlinear data can not.

The \emph{loss function} $J(\theta)$ of a classifier measures the amount of error the classifier incurs in for a particular choice of parameters $\theta$. There are various ways to formulate this function. A \emph{least-squares loss function} for a binary classifier (such as logistic regression) is presented in Equation~\ref{eq:LossFunction}
\begin{equation}
	J(\theta) = \frac{1}{2m}\sum_{i=1}^m(h_\theta(x^{(i)}) - y^{(i)})^2
	\label{eq:LossFunction}
\end{equation}
where $m$ is the number of training examples, $y \in \{0,1\}$ is the real class of the example $x$ and $h_\theta(x) \in \mathbb{R}$ is the output of the classifier for input $x$ with parameters $\theta$, this represents the probability that $x$ belongs to the positive class 1. We introduce another (rather more complex) loss function in the next section.

A classifier is trained by choosing the parameters $\theta$ that minimize its loss function, hence, minimizing the expected error of the classifier on the training set. \emph{Gradient descent} is a method used to estimate the parameters that minimize $J(\theta)$: at the start, it initializes parameters at random and iteratively updates each parameter using the gradient of the loss function until it converges to a minimum. Gradient descent is guaranteed to converge to a global minimum if the loss function is convex, convexity of the loss function depends on the model $h(x)$.

Model selection, selecting the best model $h(x)$ for a particular problem, equivalently, selecting the best classifier for the problem, is often done via cross validation. \emph{Cross validation} is an statitistical model evaluation technique where each model is trained on a subset of the data set and later validated on a disjoint subset. In \emph{hold-out cross validation} the data set is split into a training set (usually 70-90\%) and a cross validation set, each model is trained using the training set and evaluated on the cross validation set and the model which shows the best performance is selected. \emph{k-fold cross validation}, on the other hand, divides the data set in $k$ disjoint subsets (usually 5 or 10) and uses $k-1$ subsets to train the model and the remaining subset for evaluation, this process is repeated $k$ times for each model leaving out a different subset each time and the $k$ performance measures are averaged to obtain a final measure for the model. Cross validation is also used to select the \emph{model hyperparameters}, parameters that modify the underlying model, for instance, to select the number of hiddden layers in a convolutional network.

The model representation $h(x)$ needs to be chosen carefully. If we have an overly \emph{flexible} model, i.e, $h(x)$ is a complex function with many parameters to be learned compared to the size of the training set, the classifier will probably \emph{overfit} the data, this means that the parameters are fitted way too closely to the data and will pick up every small fluctuation and noise in the training set causing the trained classifier to produce almost perfect results on the training set but perform poorly on previously unseen examples. The opposite is also true, when $h(x)$ is very simple the classifier lacks the power to model the function of interest and we say that it \emph{underfits} the data. This problem is sometimes referred as the \emph{bias-variance tradeoff}. A high variance classifier is prone to overfitting, while a high bias classifier is prone to underfitting.

A popular way to avoid overfitting (and underfitting) is to use a flexible model trained with regularization. \emph{Regularization} modifies the loss function to include a penalty to the complexity of the model, thus forcing the learning stage to choose parameters that minimize both the training error of the classifier and the complexity of the model. Equation~\ref{eq:l2NormRegularization} shows the least-squares loss function with \emph{$l_2$-norm regularization}:
\begin{equation}
	J(\theta) =  \frac{1}{2m}\sum_{i=1}^m(h_\theta(x^{(i)}) - y^{(i)})^2 + \frac{\lambda}{2m} ||\theta||_2
	\label{eq:l2NormRegularization}
\end{equation}
where $||\cdot||_2$ is the euclidean norm of a vector. In addition to reducing training error, minimizing the regularized loss function will shrinken the parameters $\theta$ hopefully setting some of them to zero, thus simplifying $h(x)$. The \emph{regularization hyperparameter} $\lambda$ regulates the tradeoff between less training error and less regularization error. \emph{$l_1$-norm regularization} or \emph{lasso} is similar to $l_2$-norm regularization except that it shrinks the $l_1$-norm of $\theta$ instead of the $l_2$-norm.

To evaluate the performance of a classifier we use a separate set of examples (a test set) which should have not been used for training or cross validation. Classification accuracy is the standard performance measure in machine learning. \emph{Accuracy} measures the proportion of test set examples which are correctly classified. Its compliment, \emph{error rate}, measures the proportion of test set examples which are incorrectly classified. Accuracy, nonetheless, is not a good evaluation metric for unbalanced data sets, data sets which have many more examples of one class than the other e.g., cancer data sets are often unbalanced as most examples belong to the negative class (no cancer) than the positive class (cancer). For instance, a classifier which always predicts no cancer regardless of the input will show a high accuracy (equivalently a low error rate) even though it is not a good model for the problem.

A different set of metrics based on the confusion matrix of the classifier are used to evaluate its quality in unbalanced data sets. A \emph{confusion matrix} is a matrix which summarizes the results of a classifier in the test set (see Table~\ref{tab:ConfusionMatrix}).
\begin{table}[h]
	\centering
	\begin{tabular}{cc|c|c|}
		&\multicolumn{3}{c}{\textbf{Actual class}}\\
		&&Positive & Negative \\
		\cline{2-4}
		\textbf{Predicted}&Positive&True Positives (TP)& False Positives (FP)\\
		\cline{2-4}
		\textbf{class}&Negative&False Negatives (FN) & True Negatives (TN)\\
		\cline{2-4}
	\end{tabular}
	\caption{Confusion matrix for a binary classifier}
	\label{tab:ConfusionMatrix}
\end{table}
\emph{True positives} is the number of positive examples which were correctly predicted as positive. \emph{False positives} is the number of negative examples which were incorrectly predicted as positive. True negatives and false negatives are defined in a similar fashion. Based on the confusion matrix we can compute some commonly used metrics:
\begin{equation}
	Sensitivity \text{ or } Recall = \frac{TP}{TP+FN}
\end{equation}
\begin{equation}
	Specificity = \frac{TN}{FP+TN}
\end{equation}
\begin{equation}
	Precision = \frac{TP}{TP+FP}
\end{equation}
Sensitivity and specificity are usually preferred to present results in medical diagnosis meanwhile precision and recall are preferred in machine learning. \emph{Sensitivity} measures the proportion of positive examples predicted as positive and \emph{specificity} measures the proportion of negative examples predicted as negative. \emph{Precision} is a measure of the proportion of examples predicted as positive which are actually positive. A good classifier will have both high sensitivity and high specificity or similarly, high precision and high recall. It is always useful to have a single metric to evaluate classifiers, for example, to choose between two models; we show two commonly used metrics in Equation~\ref{eq:F1Score} and~\ref{eq:G-mean}.
\begin{equation}
	F_1\text{ }score = 2\times\frac{Precision \times Recall}{Precision + Recall}
	\label{eq:F1Score}
\end{equation}
\begin{equation}
	G\text{-}mean = \sqrt{Sensitivity \times Specificity}
	\label{eq:G-mean}
\end{equation}

In this thesis, we will generally present results for all this metrics (precision, recall or sensitivity, specificity, $F_1$ score and G-mean). The metric used when selecting a model influences its characteristics and behaviour, hence, one should put some consideration into choosing it. We favor $F_1$ score over G-mean because it concentrates on prediction in the positive class (cancer) which is harder to predict and the class we are more interested in. Furthermore, it represents a more balanced tradeoff (an small change in precision is corresponded with a small change in recall) than G-mean where an small change in specificity can be corresponded with a big change in sensitivity.
\begin{comment} Discussion of why G-mean over F1
(Not sure about this) As a rule of thumb, using G-mean will generate models that predict more positives given that the sensitivity will greatly improve and specificity will only slightly decrease. Using $F_1$ score, the model will predict less positives as that will improve precision but only slightly decrease sensitivity. 

Why G-means? Because it is more important to obtain a low error in specificity than in precision, i .e, would you prefer a 90% in specificity or a 90% in precision?. 90% in specificity means that 10% of actual negatives (10 persons) were told they have no cancer although they actually had cancer, meanwhile 10% of expected positives(a small number, maybe 1) was said he has cancer although he doesn{t. First is worse.

Using G-mean i will predict more positives, no matter what. My sensitivity is going to vastly improve and the specificity will only decrease a little, but the precision is gonna take a hit. Because if I predict less positives sensitivity is gonna go down, specificity is gonna go up (as I add more true negatives) but just a little and precision  would go up (but it wouldn't matter for g-mean).

Using F1 I'll probably predict less positives, sensitivity is going to go slightly down, and precision is going to go up, specificity doesn{t matter but it will decrease a little. Or I'll probably predcit as many poositives as needed. It focuses more on the positive class.

Other diagonal, the algorithm will learn negatives pretty well, so the one that predicts less positives(f1) probably isn{t learning much (it is predicting all negative).
\end{comment}

This section is meant to be a compendium of basic concepts in machine learning, practical machine learning involves many subtleties and implementation details not mentioned here. Notation and content in this section is mostly based on materials from Stanford's Machine Learning course\cite{Ng2014}.

%The main goal of this work is to succesfully apply convolutional networks to detect and diagnose breast cancer signs, microcalcifications and breast masses, in mammograms.
The main goal of this work is to succesfully apply convolutional networks in digital mammograms to detect and diagnose breast cancer lessions, microcalcifications and breast masses, and to compare our results to those obtained by other groups working in convolutional networks for breast cancer diagnosis.

Particularly, there are various subgoals which we expect to achieve as the project advances:
\begin{itemize}
	\item Develop a working pipeline for processing the mammographic images from our database and training a convolutional network. Essentially, this tool could also be used for other image classification tasks.
	\item Kickstart the research on deep learning in the institution.
	\item Use a simple convolutional network to perform detection and diagnosis and study these initial results to guide further research.
	\item Show the viability of convolutional networks for breast cancer diagnosis.
	\item Use convolutional networks on an entire mammogram instead of only on small patches.
	\item Improve the performance of convolutional networks reported on the literature.
	\item Generate results that could result in a conference or journal article.
	\item Propose new ideas and methods for future research in the topic.
\end{itemize}
Initial exploratory research has not yet been performed and some of these particular objectives may be modified as the project progresses. Furthermore, some new research avenues could be taken if they seem promising, for instance, using convolutional networks with digital tomosynthesis images (3-dimensional x-ray images of the breast).


\begin{comment}
Especificar en esta sección qué es lo que quiere lograr con respecto al problema identificado
en forma general y particular. Puede incluir alcances y cualquier otro
elemento que considere pertinente para delimitar su trabajo. 

{\bf Por ejemplo:}

El objetivo general de este trabajo.......

Los objetivos particulares a cumplir en este trabajo de investigación son los
siguientes: 
\begin{itemize}
	\item El primer objetivo...
	\item El segundo objetivo...
\end{itemize}

Esta sección puede contener también el {\it Modelo Particular}, que es el modelo de solución propuesto para el
problema y que obviamente debe ser consistente con los objetivos
establecidos. Se le llama {\it Modelo Particular}
 porque es en el cual se guía
el trabajo de investigación y que desemboca en lo que es la
 {\bf CONTRIBUCIÓN PERSONAL}.
Aquí es donde los aspectos de creatividad e innovación deben verse aplicados a nuestro
trabajo.
\end{comment}

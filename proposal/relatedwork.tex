In this section we offer a summary of some of the first work on using convolutional networks for breast cancer diagnosis as well as some of the articles that have influenced this thesis.

Lo et al.~\cite{Lo1995} were the first group to use convolutional networks for breast cancer detection. They used a CNN with two hidden layers to detect microcalcifications. A high sensitivity image processing technique was used to obtain a set of 2104 patches (16 by 16 pixels) of all potential disease areas from 68 digital mammograms; of these, 265 were true microcalcifications and 1821 were ``false subtle microcalcifications". Prior to training the CNN, a wavelet high-pass filtering technique was used to remove the background of these images. Each image was flipped over (left-right) and 4 rotations for each the original and flipped images were used for training (0°, 90°, 180° and 270°). The CNN was composed of one input unit ($16\times16$), 12 units in the first hidden layer ($12\times12$), 12 units in the second hidden layer($8\times 8$) and two output nodes(one for YES and one for NOT). The input size ($16\time16$), number of hidden layers ($2$) and kernel size ($5\times5$) was obtained via cross validation, altough not many other options were explored: they tried input sizes of 8, 16 or 32, one or two hidden layers and kernel sizes of 2, 3, 5 or 13. The CNN reached 0.87 average AUC when identifying individual microcalcifications and 0.97 AUC for clustered microcalcifications. Only a minimum of three calcifications was considered a detection. Sensitivity and specificity test results were not reported. This article proved that simple convolutional networks can be efficiently used for medical image pattern recognition.
%Lesson learned: two hidden layer newtwork produces better results, background reduction is neccesary and using matrices invariance to augment the data helps. Convnets work helps and convnets work.

%A refined approach was presented some years later by the same authors along some non convolutional neural networks~\cite{Lo1998}. The setting is very similar but ..... Results are..

%work done at Tec

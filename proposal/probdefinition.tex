Breast cancer is the most commonly diagnosed cancer in woman and its death rates are among the highest of any cancer. It is estimated that about 1 in 8 U.S. women will be diagnosed with breast cancer at some point in their lifetime.~\cite{CSR2014} Early detection is key in reducing the effects of breast cancer: a detection in its earlier stage (in situ) allows for better treatment and increases the survival rate to virtually 100\%.~\cite{CSR2014}

With current technology, a high quality mammogram is ''the most efective way to detect breast cancer early`` \cite{MammogramFactSheet2014}. Mammograms are x-ray images of each breast used by radiologists to search for early signs of cancer such as tumors or microcalcifications. 



90\% of breats cancers can be detected using  ammomgram and radiologist can detect up to .Nonetheless, these tasks is not easy and requires training an dexperience to be realized effectuvely  a computer aided diagnostic tool that can automatically detect these signs from digitizes images of the mammogrmas couls save the time to expert and increase the aomoun t of cancers detect or pint to experst whthe zones that could be treated woith more rcareExepret radiologists can detect up to ... but normally you do double duty... 
Radiologists normally get this amunt of time right. and a second decision would increase the amount of positives(?) detected and save time for the experts.

 
In this work we will focus on using mammograms
The project developed here has already dealed with this problem using .... and .... cite. but we tried to use convnets so that we can improve the results obtained here and in the literature with some advanced image pattern algorithms. A further review was given on t eintroduction.

What is the problem/limitations? Not yet efficient, requireds handcrafted fuigures and a lot of parameter fitting an estimation to get the better results. It depends on a lot of steps and thus could be prone to errors and requires work in various differents fields (image processing, radiology, macjhine learning) to prouce good results. Furthermore, some of tit depends on expert information (like the shapes of microcalcifications and masses and important image features) to produce results while a better systmem could let the computer figure out what are those neccesary features  

\begin{comment}
El {\it Problema} es el núcleo de la propuesta. En esta parte se define y se
justifica clara y ampliamente la situación que se pretende
resolver. Normalmente el problema particular a resolver cae dentro de un
contexto más amplio, dentro de una situación problemática de la cual se
derivan regularmente más de un problema. Los aspectos a considerar en esta
parte consisten en lo siguiente:
\begin{itemize}
	\item Describir la Situación problemática, es decir, identificar los
	problemas o áreas de oportunidad donde se ubica su investigación y los
	antecedentes de esa situación. 
	\item Definir el problema a detalle con sus factores, aspectos, relaciones
  y desarrollar la importancia de ese problema. Debería estar basado en
  literatura.
\end{itemize}
\end{comment}
